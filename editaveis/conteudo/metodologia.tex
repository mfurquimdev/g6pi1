\chapter[Metodologia]{Metodologia}

	A metodologia que será usada no seguinte projeto é o Scrum, referente a metodologia ágil. A escolha dessa metodologia teve o intuito de tornar o trabalho mais fluido, dando mais dinamicidade e agilidade na confecção do trabalho. Outro benefício dessa metodologia é a facilidade de implementar mudanças no trabalho para atender as exigências do Dono do produto.
	
	O nosso Product Backlog será baseada no nosso escopo e na nossa EAP. Nele haverá uma lista de funcionalidades que o nosso produto deve conter. Ele será de suma importância pois, a partir dele, será gerado o material para ser trabalhado nas Sprints.
	
	No decorrer do projeto haverá ciclos, Sprints, que terão a duração de uma semana. E em cada Sprint terá o desenvolvimento de atividades irão compor o produto final, o projeto da casa autossustentável. Elas começarão toda segunda feira com a Sprint Planning Meeting, em que nela será descrito os requisitos de maior importância e que devem estar na Sprint. Esses requisitos darão origem ao Sprint Backlog. E por fim elas terminam todo domingo. 
	
\section{Grupos e divisões}

	A equipe foi dividida em três grupos de pesquisa e um grupo de comunicação e documentação, para melhor organizar as pesquisas e o desenvolvimento do trabalho. Cada grupo possui um sub-gerente para concentrar a comunicação e não haver redundância de arquivos ou desentendimento de tarefas. 

\subsection{Gerente}
\begin{itemize}
\item \textbf{Guilherme Silva Lionço}
\end{itemize}

Foram selecionadas um grupo de pessoas para cuidar da parte de documentação e comunicação do time.
\subsection{Documentação e Comunicação}
\begin{itemize}
\item \textbf{Mateus M. F. Mendonça}
\item Vitor N. A. Ribeiro
\end{itemize}

Os grupos de pesquisa que foram divididos são:

\subsection{Estrutura e Materiais}

\begin{itemize}
\item \textbf{Igor A. L. da Costa}

\item Joao G. S. M. de Paula

\item Kleiton N. Silva

\item Klyssmann H. F. de Oliveira

\item Marcelo M. de Oliveira

\item Tulyane M. dos Santos

\item Vitor N. A. Ribeiro

\item Yuri S. Alves
\end{itemize}

\subsection{Eficiência Energética}

\begin{itemize}
\item \textbf{Jessica K. O. de Sousa}

\item Barbara C. Ferro

\item Ivson A. Rocha

\item Jose M. M. F. Junior

\item Stefany S. Aquino

\item Talyta V. Cabral

\item Walter L. Baldez
\end{itemize}

\subsection{Automação}

\begin{itemize}
\item \textbf{Gabriel S. Soares}

\item Arnoldo T. M. Lima

\item Flavio C. Paixão

\item Itiane T. B. Almeida

\item Lucas V. T. Brilhante

\item Marcelo M. de Oliveira

\item Victor B. Batalha\\
\end{itemize}

E foi criado um sub-grupo de \textbf{Software} dentro de \textbf{Automação}.

\subsection{Software}

\begin{itemize}

\item \textbf{Vitor N. A. Ribeiro}

\item Flavio C. Paixão

\item Lucas V. T. Brilhante

\item Marcelo M. de Oliveira

\item Mateus M. F. Mendonça
\end{itemize}
