\chapter{Plano de aquisição}

Este documento visa orientar a aquisição de S\&SC
para < objetivo esperado do S\&SC> da < nome da empresa >.

\section{Objetivo da aquisição}
(Descrição dos objetivos a serem atendidos com a aquisição do S\&SC).
Exemplo: Pretende-se, com a aquisição do S\&SC, controlar as finanças da
instituição, de forma a agilizar os processos administrativos, aliviando a alta carga
de trabalho da tesouraria, melhorando e dinamizando as rotinas administrativas e os
controles financeiros; e melhorar a qualidade das informações gerenciais;

\section{Requisitos}
\subsection{Requisitos dos interessados (stakeholders)}

(Lista de necessidades dos interessados (stakeholders) na utilização do
software a ser adquirido. Considerar os diversos stakeholders e contextos do uso
software. A definição de prioridades pode ser importante para estabelecer critérios
de aceitação e plano de versões do software. Eventualmente esta relação de
requisitos pode se constituir num documento anexo ao plano de aquisição).

Exemplos:
Agilizar os processos administrativos.
Amenizar a alta carga de trabalho da tesouraria.
Permitir o controle das contas a receber.

\subsection{Requisitos do sistema}
(Descrição do contexto geral no qual o software a ser adquirido estará
inserido, podendo contemplar ambiente tecnológico, de processos e até mesmo de
pessoas envolvidas).
Exemplo: O software deve trabalhar em rede de microcomputadores e
ambiente Windows, de maneira a aproveitar a infra-estrutura existente, utilizando o
banco de dados FireBird, que é o banco corporativo da organização. O software
será um dos componentes do processo de aquisição de insumos da empresa,
contemplando as atividades “a”, “b” e “c”.

\subsection{Requisitos do software}
(É a derivação dos requisitos dos interessados (stakeholders) que foram
mapeados através dos sistemas. Os requisitos do software dividem-se em
Requisitos Funcionais que descrevem as funções a serem realizadas pelo software
a ser adquirido e Requisitos de Qualidade que descrevem as características de
qualidade consideradas importantes no software).
Exemplos de requisitos funcionais:
O software deverá permitir cadastrar usuário com seu grau de sigilo.
O software deverá permitir redigir documento.
O software deverá permitir visualizar documento.
Exemplos de requisitos de qualidade:
Usabilidade: estilo ou princípios de diálogo que são aplicáveis; tipo
de documentação a ser entregue (on-line, manuais de usuário);
Portabilidade: Regras de portabilidade que deverão ser adotadas
(tanto para a parte de servidores quanto para acesso via estações de
trabalho);
Interoperabilidade: integração das aplicações novas com os bancos
de dados e aplicações legadas;
Manutenibilidade: tipos e características dos artefatos gerados, de
modo a permitir a manutenção por parte do contratado, bem como para facilitar
eventuais repasses de conhecimento.

\subsection{Requisitos de projeto}
(Estabelecer o ciclo de vida de desenvolvimento a ser adotado, técnicas,
ferramentas, tecnologias, métodos, forma de gestão e de documentação).
Exemplo: O software a ser adquirido deverá ser desenvolvido segundo a
abordagem do Processo Unificado, gerando artefatos segundo a notação
UML e com tecnologia J2EE.

\subsection{Requisitos de manutenção}
(Estabelecimento da forma como será conduzida a manutenção do
software a ser adquirido. Definir o custo e o canal de comunicação entre o
fornecedor e o cliente para o atendimento de possíveis problemas).
Exemplo: A correção de problemas considerados críticos deverá ser
providenciada em até 24 horas após a sua identificação pelo usuário, ou, não
sendo viável, deverá ser estabelecida uma solução de contorno; A cada 2
anos deverá ser promovida uma atualização tecnológica do software
considerando as evoluções que ocorrerem no seu ambiente de operação.

\subsection{Requisitos de treinamento}
(Estabelecimento de um plano de treinamento para a operação do
software a ser adquirido. Definir as pessoas que participarão do treinamento, o
número de apresentações/aulas que serão necessárias assim como o material e o
ambiente a ser utilizado).
Exemplo: A organização fornecedora deverá oferecer 3 apresentações/aulas
aos usuários do software. Deverá fazer parte do material de treinamento o
manual do usuário. O treinamento será realizado nas dependências da
organização cliente.

\subsection{Requisitos de implantação}
(Estabelecimento da forma como será conduzida a implantação do
software a ser adquirido. Definir o ambiente e os equipamentos necessários).
Exemplo: A implantação do software será realizada em 3 dias. A organização
fornecedora deverá acompanhar as instalações dos novos equipamentos e do
novo software. Ao se implantar o software, o banco de dados deverá estar
preenchido com os dados reais.

\section{Termos contratuais}
(Descrição de aspectos relacionados ao contrato).

\subsection{Tipo de contrato a ser empregado}
(Tipo de contrato a ser utilizado)
Exemplo: Contrato de preço fixo, contrato de custos reembolsáveis ou
contrato de preço unitário por ponto por função.

\subsection{Multas e penalidades}
(Valor e as condições de ocorrência de multas de ambas as partes).
Exemplo: A contratada, ressalvados os casos fortuitos ou de força maior,
devidamente comprovados, e garantida a sua prévia defesa no respectivo
processo de apuração dos fatos, estará sujeita às seguintes penalidades:
a. advertência, por escrito, na primeira falta cometida;
b. multas, no valor de até 20\% do valor total estabelecido;
c. suspensão temporária de participação em licitação e impedimento de
contratar com o cliente, por prazo de até dois (02) anos.

\subsection{Direitos de distribuição, uso e propriedade do software}
(Estabelecimento dos direitos de distribuição, uso e propriedade do
software, como, por exemplo, o número de cópias a serem distribuídas e a
propriedade do código fonte, entre outros).
Exemplo: O software desenvolvido estará sob uma licença de uso restrito ao
contratante, protegidos por direitos autorais e de propriedade. A cópia,
redistribuição, engenharia reversa e modificação do software proprietário são
proibidas. Os programas de software serão de uso proprietário da
organização cliente, inclusive seus códigos-fonte e documentação. A
organização fornecedora não tem direito, disponibilidade ou qualquer outro
tipo de participação em nenhum destes programas ou em qualquer cópia,
modificação ou parte agregada de qualquer um destes programas.

\subsection{Garantia do S\&SC}
(Garantia do S\&SC descrevendo o prazo de validade, a abrangência
(por exemplo, erros no software, problemas na instalação, documentação,
integração com outros sistemas) e os procedimentos para o seu uso).
Exemplo: Durante o prazo de garantia, de seis meses, a contratada deverá
prestar serviços de manutenção, esclarecendo dúvidas e corrigindo eventuais
falhas que impossibilitem o uso normal do software.

\section{Termos financeiros}
(Descrição de questões financeiras relacionadas à aquisição).

\subsection{Orçamento do projeto}
(Valor monetário disponível para o projeto de aquisição).
Exemplo: O valor disponível para a aquisição do software é de R\$
1.000.000,00 (Um milhão de reais).

\subsection{Fonte de recursos para a aquisição}
(Descrição da origem da verba alocada para a aquisição).
Exemplo: A verba para o projeto de aquisição é fruto de uma parceria com
organizações afins.

\subsection{Formas de pagamento da aquisição}
(Descrição dos períodos de pagamento ao fornecedor, o número de parcelas
o valor de cada parcela).
Exemplo: O pagamento referente à aquisição será realizado em quatro
parcelas no valor de R\$250.000,00 (duzentos e cinquenta mil reais) cada, ao
longo de um período de um ano.

\section{Termos técnicos}
(Descrição de aspectos técnicos considerados importantes para a aquisição).

\subsection{Procedimentos de confidencialidade}
(Estabelecimento do tratamento que deve ser dado às informações
sigilosas confiadas ao fornecedor, bem como as condições de acesso às
instalações do adquirente, identificação dos participantes do projeto, entre
outros).
Exemplo: É de responsabilidade do fornecedor proteger e devolver toda e
qualquer documentação sigilosa emprestada pela organização cliente durante
a elaboração do S\&SC. O fornecedor deverá eleger um responsável pelo
pedido, guarda e devolução dos documentos necessários durante a
aquisição.

\subsection{Especificação do canal de comunicação}
(Estabelecimento de um mecanismo de comunicação entre os
participantes do projeto de aquisição e o fornecedor: via e-mail, pessoalmente
ou por telefone,sempre que houver necessidade).
Exemplo: Sempre que houver necessidade, a troca de informações entre a
organização cliente e o fornecedor poderá ser realizada via e-mail e/ou
pessoalmente. Tanto os e-mails trocados quanto as reuniões presenciais
devem ser registrados e armazenados.

\subsection{Procedimentos para mudanças}
(Estabelecimento de como, quando e por quem serão executadas as
alterações nos requisitos e no contrato).
Exemplo: Tanto a organização cliente quanto a organização fornecedora
deverão eleger um responsável pela gerência de pedidos de alteração de
requisitos e de contrato. Sempre que houver a necessidade de alguma
mudança, os representantes responsáveis deverão se reunir e chegar a um
acordo sobre a realização ou não da alteração em questão.

\subsection{Procedimentos para tratamento de problemas}
(Procedimentos a serem adotados para registro, acompanhamento e
solução de problemas).
Exemplo: À medida que sejam identificados problemas que possam afetar os
resultados do projeto para o adquirente, esses deverão ser registrados, ter
seus impactos analisados e encaminhamentos da solução definidos, incluindo
os responsáveis pelas ações a serem tomadas, os prazos envolvidos e data
da efetiva solução.
Problemas no âmbito técnico específico dos projetos e que não afetem os
resultados para o adquirente deverão ser tratados pelos procedimentos
internos de tratamento de problemas do fornecedor.

\section{Lista de S\&SC a serem entregues}
(Lista dos S\&SC que devem ser entregues pelo fornecedor no final do contrato.
Entre eles, devem ser considerados os serviços de suporte esperados do
fornecedor). Exemplo: Os produtos a serem entregues ao final do contrato são:
(i) o software instalado em seu ambiente de operação;
(ii) os manuais do usuário, de instalação e do sistema; e
(iii) os códigos-fonte

\section{Pontos de controle}
(Descrição dos marcos de controle do projeto, definidos por meio dos produtos
de trabalho e dos processos do fornecedor que serão avaliados pelo adquirente
durante o processo de aquisição, e o método de avaliação, por exemplo: validação,
auditoria e revisão conjunta, entre outros).
Nome
do
Produto/Processo
Módulo
1
manutenção
dados do BD
Manual do usuário
Método da Avaliação
– Testes
dos
Revisão Conjunta

\section{Prazos estabelecidos}
(Especificação do cronograma para o ciclo de vida escolhido e seus marcos).
Exemplo: O software a ser adquirido é composto por quatro módulos. O primeiro
módulo (xxxx) deverá ser entregue, para testes do cliente, ao final de dois meses, a
contar da data da assinatura do contrato. O segundo módulo (yyyy) deverá ser
entregue três meses após a entrega do primeiro. O terceiro módulo (zzzz) deverá ser
entregue três meses após a entrega do segundo módulo e o quarto e último módulo
(wwww) deverá ser entregue quatro meses após a entrega do terceiro módulo.

\section{Critérios de seleção do fornecedor}
(Descrição dos critérios a serem avaliados para julgamento da capacidade do
fornecedor em atender ao contrato pretendido).
Exemplo: Os critérios para a seleção do fornecedor são:
(i) Situar-se na cidade do Rio de Janeiro;
(ii) Ter mais de cinco anos de mercado;
(iii) Ter experiência no domínio do problema; e
(iv) Ter avaliação oficial MA-MPS nível F

\section{Critérios de aceitação do S\&SC}
(Descrição de aspectos que devem ser satisfeitos para que o S\&SC sejam
aceitos. Teoricamente todos os requisitos teriam que ser avaliados, o que nem
sempre é prático. Estes são critérios que serão avaliados para apoiar o processo de
aceitação. A garantia pode assegurar que os demais requisitos terão que ser
atendidos durante o seu prazo de vigência).

\subsection{Requisitos funcionais do software}
(Descrição das funções do software que serão avaliadas para a definição
de sua aceitação).
Exemplo: O software só será aceito após a validação com sucesso das
funções de cadastramento, cálculo e consultas gerenciais.

\subsection{Requisitos de qualidade do software}
(Descrição das características de qualidade que serão avaliadas para a
definição da aceitação do software).
Exemplo: O software só será aceito após avaliação com sucesso dos
requisitos referentes às seguintes características de qualidade:
(i) segurança de acesso;
(ii) usabilidade;
(iii) comportamento em relação ao tempo; e
(iv) portabilidade

\subsection{Documentação disponível}
(Especificação dos documentos que serão avaliados como condição de
aceitação do S\&SC, como: manual do usuário e de instalação, entre outros).
Exemplo: O software a ser adquirido deverá ser entregue juntamente com o
manual do usuário, manual do sistema e manual de instalação.

\section{Normas e modelos}
(Descrição de normas, modelos, leis, padrões, práticas e convenções que devem
ser seguidos pelo fornecedor).
Exemplo: A organização fornecedora deverá seguir o modelo MPS.BR para o
desenvolvimento de software e as normas adotadas pela organização cliente com
relação a padronização da nomenclatura de variáveis dos programas de software.

\section{Responsabilidades do projeto}
(Definição das tarefas a serem desempenhadas no projeto, considerando o
adquirente, o fornecedor e, quando houver, terceira-parte).
Exemplo: A equipe do projeto de aquisição da organização cliente deverá fornecer,
sempre que necessário, informações e documentos que serão utilizados pela
organização fornecedora. Fica também sob a responsabilidade da equipe do projeto
de aquisição da organização cliente a atividade de prover as informações
necessárias para o preenchimento do banco de dados. Além das atividades típicas
do fornecedor, previstas no plano de projeto, deverá executar funções de gerente de
projeto, que atuará de forma global no projeto, assegurando que as ações sejam
tomadas de forma adequada e a tempo para atender às necessidades de projeto;

\section{Riscos e eventos}
(Descrição de riscos ou eventos que podem ocorrer durante a aquisição e como
devem ser tratados).

\subsection{Identificação do risco}
(Descrição do tipo de risco, por exemplo: atraso no cronograma, falta
de recursos financeiros e humanos e falha de interpretação dos requisitos do
software, entre outros).
Exemplo: Um risco que pode ocorrer durante a execução do projeto é a
complexidade de requisitos.

\subsection{Probabilidade de ocorrência}
(Descrição da probabilidade do risco ocorrer, por exemplo: alta, média
ou baixa).
Exemplo: A probabilidade de ocorrência do risco identificado no item 13.1 é
alta.

\subsection{Impacto no projeto}
(Descrição dos aspectos relevantes que podem afetar o projeto caso o
risco ocorra, por exemplo: parar o projeto e falta de verbas para outras
atividades, entre outros).
Exemplo: Os impactos no projeto decorrentes do risco identificado no
item 13.1 são: o alto índice de alteração dos requisitos e cronograma
ultrapassado.

\subsection{Mitigação dos riscos}
(Descrição dos procedimentos para amenizar ou eliminar a ocorrência
do risco).
Exemplo: Para mitigar o risco identificado no item 13.1 será consultado um
especialista no domínio do problema para esclarecer dúvidas e orientar a
atividade elicitação de requisitos do software.

\subsection{Plano de contingência}
(Descrição dos procedimentos a serem tomados no caso do risco
se concretizar).
Exemplo: Caso o risco identificado no item 13.1 se concretize, poderá ser
necessário considerar uma nova abordagem para complementação dos
requisitos e para desenvolvimento do software, podendo adotar, por exemplo,
a aplicação de protótipos.


\chapter{Aquisição e a Engenharia de Software baseada em componentes}
Uma possível abordagem para aquisição de software está relacionada à utilização
de componentes. Ainda que seja possível personalizar o processo descrito neste
guia, visando a aquisição de componentes de software, é importante que se levem
em conta as considerações a seguir.
De acordo com [SAMETINGER,1997], o componente de software deve: ser
autocontido, ser identificável, ter funcionalidade (descrever e/ou desempenhar
funções específicas), ter interfaces, ter documentação (indispensável para o reúso) e
ter status de reúso definido.
Outra definição, que auxilia o entendimento do componente como um produto, numa
visão comercial, sendo desenvolvido por produtores (fornecedores) e adquirido por
consumidores (desenvolvedores), como um artefato para composição de outros
sistemas, é colocada por [D’Souza, 1998]: “um componente (código) é um pacote
coerente de implementação que pode ser desenvolvido e distribuído
independentemente, provê interfaces explícitas e bem especificadas, define
interfaces que ele precisa de outros componentes, e pode ser combinado com
outros componentes pela configuração de suas propriedades, sem a necessidade de
modificação”.
Um novo desafio para o desenvolvimento baseado em componentes é o chamado
problema da confiança no componente. O problema está na dificuldade em saber se
o componente faz o que realmente deveria fazer. Um consumidor muitas vezes
precisa decidir entre vários componentes, de fornecedores distintos, por aquele
adequado à composição do novo sistema.
O fornecedor de um componente pode medir as características de qualidade de um
componente como uma unidade, mas não pode prever todos os seus futuros
contextos de reutilização e garantir sua adequação. Esta questão é especialmente
mais difícil quando se trata de componentes desenvolvidos por terceiros ou de
componentes do tipo COTS - Commercial Off-The-Shelf (desenvolvidos
comercialmente, disponíveis em prateleiras), os quais normalmente são distribuídos
sem o código fonte. Todavia, esta questão também está presente para componentes
desenvolvidos na própria organização, onde a falta de documentação e a dificuldade
de comunicação das equipes de desenvolvimento podem produzir um efeito similar.
A garantia de sucesso do desenvolvimento baseado em componentes depende da
qualidade dos componentes de software. Os desenvolvedores precisam saber se o
componente é confiável e adequado ao sistema [VOAS, 1998]. Muitos componentes
de software são oferecidos como “caixas-pretas”, como objetos executáveis, aos
quais as licenças não permitem acesso aos códigos fontes (ou licenciados a preços
proibitivos). Então, como saber se um determinado componente é adequado para
integrar um sistema baseado em componentes? Como prever se realizará a função
necessária ao encaixar-se na arquitetura? Ou, ainda, se preenche os requisitos
desejados com a qualidade adequada?
Se considerarmos componentes como pacotes de software, é possível utilizar os
conceitos contidos na NBR ISO/IEC 12119 8 que trata dos seus requisitos de
qualidade e teste. Estes conceitos podem ser de grande valia para os consumidores
de componentes ou pacotes de software. Eles tratam de aspectos importantes que
podem ser abordados durante sua aquisição. Portanto, ao adquirir um componente
ou pacote de software o consumidor deve verificar:
1 - Descrição do produto: um documento contendo as propriedades e o
principal propósito do pacote de software ou componente, ajudando o
comprador na avaliação de sua adequação antes de adquiri-lo. Esta
descrição deverá conter:
Identificação única do documento,
Funcional ou Informações do Produto;
Identificação do produto contendo pelo menos o nome e a sua
versão ou data;
O nome e endereço de pelo menos um dos fornecedores;
Tarefas que podem ser realizadas pelo produto;
Requisitos do sistema, como: unidade de processamento, tamanho
da memória principal, tipos e tamanhos de armazenamentos
periféricos, equipamentos de entrada e saída, ambiente de rede,
software do sistema operacional e outros tipos de software;
Interfaces com outros produtos;
Itens a serem entregues;
Informação de instalação (se pode ou não ser executada pelo
comprador);
Informação de suporte (se haverá ou não suporte para a operação
do produto);
Informação de manutenção (se haverá ou não manutenção do
produto e o que será incluído nela);
Visão geral das funcionalidades do produto, os dados necessários e
as facilidades oferecidas;
Valores limites suportados pelo produto;
Informações de segurança para prevenir acesso não autorizado a
programas ou dados;
por
exemplo:
Descrição
8
Esta norma foi atualizada pela norma internacional ISO/IEC 25051, porém os conceitos relacionados
nesta seção são semelhantes e foram mantidos para manter a compatibilidade com o texto
relacionado às referências bibliográficas citadas.
Informações de confiabilidade, como: verificar se as entradas são
plausíveis, proteção contra erros de usuários e recuperação de
erros;
Informações de usabilidade, como o tipo de interface usada com o
usuário, conhecimento requerido para o uso do produto, se o
produto pode ser adaptado pelo usuário e em que condições, e
procedimentos usados para a proteção contra cópias;
Informações relativas à eficiência do produto, como tempo de
resposta;
Informações quanto à manutenibilidade 9 ; e
Informações quanto à portabilidade 10 .
2 - Documentação do usuário: um documento contendo as informações
necessárias para o uso do produto. Todas as funções citadas na descrição
do produto e suas formas de acionamento pelo usuário devem estar
descritas neste documento. Este documento deve ser completo, correto,
consistente e inteligível.
3 – Informações relativas a programas e dados: se a instalação puder
ser realizada pelo comprador então é necessário um manual de
instalação. Os programas e os dados devem corresponder e não
apresentar contradições com a descrição do produto e à documentação do
usuário.
4 – Instruções para teste: estas instruções descrevem como um produto
deve ser testado com relação aos requisitos de qualidade. Estes testes
incluem tanto o teste para as propriedades requeridas quanto o teste para
as propriedades prometidas pela descrição do produto. Eles incluem o
teste de inspeção dos documentos que fazem parte do produto como:
descrição do produto, documentação do usuário, programas e dados,
assim como os testes de caixa-preta para avaliar o comportamento de
seus programas e dados.
A qualidade de componentes de software é fundamental para o sucesso de
aplicações baseadas em componentes.
Segundo [Simão, 2002], as características e subcaracterísticas de qualidade
abordadas pela NBR ISO/IEC 9126-1 também podem ser utilizadas como metas a
serem atingidas no desenvolvimento, na seleção e na aquisição de componentes.
É importante ressaltar, para aqueles que desejam adquirir componentes para
desenvolver software com reúso, a existência da IEEE Std 1517:1999 como uma
norma específica para o desenvolvimento para reúso e que é uma extensão da
ISO/IEC 12207.
9
Capacidade do produto de software de ser modificado. As modificações podem incluir correções,
melhorias ou adaptações do software devido a mudanças no ambiente e nos seus requisitos ou
especificações funcionais. [NBR ISO/IEC 9126-1].
10
Capacidade do produto de software de ser transferido de um ambiente para outro.
[NBR ISO/IEC 9126-1].


\chapter{Documento de visão}
\section{Introdução}

	O documento de visão busca coletar, analisar e definir necessidades e recursos de nível superior do \textit{Sistema de Automação para Casa Sustentável}. Concentra-se nos recursos necessários, usuários-alvo e nas razões do projeto. Os detalhes de como o SACS satisfaz essas necessidades são descritos nos casos de uso e nas especificações suplementares.

\subsection{Finalidade}

	Este documento tem como objetivo apresentar, analisar e definir as necessidades e características do SACS. O SACS tem como objetivo controlar e monitorar todo o sistema do Projeto “Casa Sustentável”. O documento descreverá as características gerais do sistema, os envolvidos em seu desenvolvimento e o público alvo.


\subsection{Escopo}

	O SACS, permite o usuário monitorar gasto de energia elétrica e consumo de água, controlar o sistema de segurança, visualizar a imagem das câmeras de segurança e controlar todo o sistema automatizado, incluindo sistema de jardinagem, luzes internas e outros aparelhos internos. O sistema também deve permitir a visualização da produção de energia das placas de energia solar e dos motores eólicos, além de mostrar o nível de reservatório de água da chuva.

\subsection{Visão Geral}

	Faremos uso de nosso documento como uma forma de organizar informações necessárias para futuras atividades do projeto em desenvolvimento. Informações a cerca do contexto do projeto, funcionalidades, e características serão organizados e estarão disponíveis neste documento afim de poupar qual vir a atrasar o processo de desenvolvimento do projeto. Todo e qualquer tempo será poupado com o uso deste documento de visão.

\section{Posicionamento}

\subsection{Descrição do Problema}

\begin{longtable}{|l|m{7cm}|}
	\hline \textbf{O problema} & Valor elevado do consumo desnecessário de energia elétrica\\
	\hline \textbf{Afeta} & Pessoas que moram em residências, sejam casas ou apartamentos, e fazem uso da energia elétrica\\
	\hline \textbf{Cujo impacto é} & Alto custo da conta de energia elétrica\\
	\hline \textbf{Uma boa solução seria} & Desenvolvimento de um sistema de automação capaz de controlar o consumo da energia elétrica de acordo com o requisitos.\\
	\hline
\end{longtable}

\subsection{Sentença de Posição do Sistema}

\begin{longtable}{|l|m{7cm}|}
	\hline \textbf{Para} & Cidadãos brasileiros residentes na Casa Sustentável e inteligente \\
	\hline \textbf{Que} & Se importam com o alto consumo desnecessário de energia elétrica e desejam maior conforto\\
	\hline \textbf{O SACS} & É um software ou um conjuntos de softwares de transmissão e controle de dados que integra sensores e age de forma inteligente\\
	\hline \textbf{Que} & Garante conforto para os residentes e economia de energia\\
	\hline \textbf{Diferente de} & Crestron\footnote{http://www.crestron.com/}\\
	\hline \textbf{Nosso Sistema} & Integra sensores de forma inteligente para controle do consumo de energia elétrica, armazenamento e reuso de água não potável\\
	\hline
\end{longtable}

\section{Descrições dos Envolvidos e Usuários}

\subsection{Resumo dos Envolvidos}

\begin{longtable}{|m{5cm}|m{5cm}|m{5cm}|}
	\hline \textbf{Nome} & \textbf{Descrição} & \textbf{Responsabilidades}\\
	\hline Professor das disciplina de Projeto Integrador 1. & É responsável por ministrar aulas de projeto integrador 1, direcionando caminhos para o projeto. & Mantem o foco do projeto ministrando aulas sobre gerência de projetos, direcionando o caminho a ser seguido pelo projeto. \\
	\hline Alunos de Projeto Integrador 1. & São os alunos que cursam a disciplina de Projeto Integrador 1. & Iram desenvolver o projeto, especificando quais tecnologias serão usadas, que se encaixam melhor no escopo do projeto, visando a sustentabilidade.\\
	\hline Aluno Gerente de Projeto Integrador 1. & É o aluno que cursa a disciplina de projeto Integrador e assumiu a posição de gerente. & Irá garantir que todos do grupo estão desempenhando sua função e deverá integrar todas as partes e passar para o gerente do projeto “Casa Sustentável”.\\
	\hline Desenvolvedores e Gerentes & Profissionais da área que implementarão e integrarão o projeto já pronto dos alunos de PI1. & Irão produzir o sistema visionado e projetado na matéria de PI1.\\
	\hline Moradores da “Casa Sustentável” & Usuário das funcionalidades da “Casa Sustentável” & Possuir fundo financeiro necessário para investir no projeto da “casa sustentável”\\
	\hline
\end{longtable}

\subsection{Resumo dos Usuários}

\begin{longtable}{|m{5cm}|m{10cm}|}
	\hline \textbf{Nome} & \textbf{Descrição}\\
	\hline Moradores da “Casa Sustentável” & Pessoa Física que adquiriu o produto “Casa Sustentável”, especificado no projeto da disciplina de PI1.\\
	\hline
\end{longtable}

\subsection{Perfis dos Envolvidos}

\textit{Equipe projetista da disciplina de Pi1}

\begin{longtable}{|m{5cm}|m{10cm}|}
	\hline \textbf{Representantes} & Arnoldo T. M. Lima\\ & Flavio C. Paixao\\ & Itiane T. B. Almeida\\ & Lucas V. T. Brilhante\\ & Marcelo M. Oliveira\\ & Mateus M. F. Mendon\c{c}a\\ & Talyta V. Cabral\\ & Victor B. Batalha\\ & Gabriel S. Soares\\ & Vitor N. A. Ribeiro\\
	\hline \textbf{Descrição} & Farão o documento do projeto com todas as especificações técnicas necessárias.\\
	\hline \textbf{Tipo} & Estudantes da Universidade de Brasília, Campus do Gama da matéria de Projeto Integrador 1, que pesquisaram todas as especificações do projeto e desenvolveram usando todos os conhecimentos de projeto passado pelos professores.\\
	\hline \textbf{Responsabilidades} & Trabalhar em conjunto para montar um documento que especifique todos os aspectos do projeto. Pesquisar conhecimentos técnicos a serem aplicados no projeto.\\
	\hline \textbf{Critério de Sucesso} & Gerir de maneira adequada, trabalhar em conjunto, desenvolver em equipe adequando ao tema do projeto, custo e prazo definido pela matéria.\\
	\hline \textbf{Envolvimento} & Alto.\\
	\hline
\end{longtable}

\textit{Gerente de automação do projeto “Casa Sustentável”}

\begin{longtable}{|m{5cm}|m{10cm}|}
	\hline \textbf{Representantes} & Gabriel S. Soares\\
	\hline \textbf{Descrição} & Aluno da matéria de PI1 que é gerente da área de automação.\\
	\hline \textbf{Tipo} & Aluno habilitado e disponível, capaz de gerir a equipe de projeto de automação.\\
	\hline \textbf{Responsabilidades} & Garantir que todos os membros estão desempenhando sua função e entregando o que foi decidido. Deve também organizar as entregas de forma a integrar os assuntos e garantir a qualidade do projeto.\\
	\hline \textbf{Critério de Sucesso} & Gerir corretamente o tempo de forma a abranger a diferentes variáveis que podem atrasar a entrega.\\
	\hline \textbf{Envolvimento} & Alto.\\
	\hline
\end{longtable}

\textit{Principais necessidades dos Usuários ou dos Envolvidos}

\begin{longtable}{|m{2.75cm}|m{2cm}|m{3cm}|m{4cm}|m{4cm}|}
	\hline \textbf{Necessidade} & \textbf{Prioridade} & \textbf{Preocupações} & \textbf{Solução Atual} & \textbf{Solução Proposta}\\
	\hline 
Visualizar o gasto e a produção energética da casa.
&
Alta
&
Obter dados energéticos para visualização do balanço, já que, o objetivo do projeto “Casa sustentável” ter ser sustentável, isto é, inclui gerar sua própria energia.
&
Não há solução.
&
Receber dados vindo da smart grid e dos geradores de energia e mostrar para o usuário através de gráficos.
\\
	\hline 
Controlar aparelhos domésticos com a automação implementada.
&
Alta
&
O usuário tem que ter a habilidade de controlar todos os componentes automatizados da casa.
&
Há vários softwares no mercado que fazem esse tipo de monitoramento no mercado, mas nenhum deles tem integração com todos os sistemas de monitoramento sustentável e de geração de energia que são necessários.
&
Desenvolver um sistema embarcado que consiga receber todos os dados de sensores e consiga controlar os aparelhos com automação.
\\
	\hline
Acessibilidade
&
Alta
&
O software deve ter uma interface intuitiva e fácil de usar, já que o software engloba muitas funções de monitoramento e controle.
&
Busca de técnicas de programação e criação de interface simples.
&
Analisar softwares presentes no mercado e fazer trabalho de pesquisa com protótipos com usuários reais.
\\
	\hline
\end{longtable}

\section{Ambiente do Usuário}

	O sistema será implementado e testado em um servidor Linux conectado através de ZWAVE juntamente ao que será o SACS.

	As restrições de uso incluem a compra do dispositivo com o software embutido e conexão com a internet na casa.

\section{Visão Geral do Sistema}

\subsection{Perspectiva do Sistema}

	O sistema tem como objetivo automatizar residências com a finalidade de trabalhar com termostatos, fechaduras, sistema de segurança, equipamentos de áudio e video, sistema de irrigação, portas de garagem, sensores ambientais, câmeras, luzes, gerenciamento de energia, internet e redes relacionadas, controlador de piscinas e spa, integração com radio frequência, integrado com robôs de limpeza, segurança dos dados, integração com telefones, fax, emails e possuir interface de usuário.

\subsection{Suposições e Dependências}

	Para a utilização do sistema é suposto que a casa possua conexão com internet e vários sensores para controle do gasto de energia e sensor de fluxo de água.
	
	Caso haja necessidade de alterações no projeto, as alterações deverão ser feitas neste documento traduzindo de melhor maneira possível.

\subsection{Licenciamento e Instalação}

	Alguns recursos explorados no desenvolvimento do sistema poderá ser necessário a compra de licenças.


\section{Requisitos Funcionais do Sistema}

	Esta seção tem a função de informar de forma resumida todas as características e funcionalidades do sistema.
	
\begin{tabular}{|l|l|c|c|}
\hline 
\textbf{Requisito} & \textbf{Descrição} & \textbf{Prioridade} & \textbf{Dependência}\tabularnewline
\hline 
\hline 
RF1 & Monitoramento energético & Alta & -\tabularnewline
\hline 
RF2 & Ambientação & Médio & RF9, RF3\tabularnewline
\hline 
RF3 & Gestão da informação & Alta & -\tabularnewline
\hline 
RF4 & Controle de cortinas & Médio & RF3\tabularnewline
\hline 
RF5 & Automatização de multimídia & Médio & RF3\tabularnewline
\hline 
RF6 & Automatização hídrica & Médio & RF3\tabularnewline
\hline 
RF7 & Automatização do Jardim & Baixa & RF3\tabularnewline
\hline 
RF8 & Segurança & Alta & -\tabularnewline
\hline 
RF9 & Detecção de presença & Médio & -\tabularnewline
\hline 
RF10 & Monitoramento de humidade & Baixa & -\tabularnewline
\hline 
\end{tabular}

\section{Requisitos Não Funcionais do Sistema}

	Resumo em alto nível de outras características do sistema, tipicamente não funcionais.

\begin{tabular}{|l|l|}
\hline 
\textbf{Identificador} & \textbf{Recurso}\tabularnewline
\hline 
\hline 
RNF1 & Taxa de transmissão de dados\tabularnewline
\hline 
RNF2 & Estabilidade do sistema\tabularnewline
\hline 
RNF3 & Integridade do sistema\tabularnewline
\hline 
RNF4 & Monitoramento dos dados\tabularnewline
\hline 
RNF5 & Precisão dos dados dos sensores\tabularnewline
\hline 
RNF6 & Usabilidade do sistema\tabularnewline
\hline 
RNF7 & Unidade central de processamento\tabularnewline
\hline 
RNF8 & Monitoramento em tempo real\tabularnewline
\hline 
\end{tabular}

\section{Procedência e Prioridades}

\begin{tabular}{|l|l|}
\hline 
\textbf{Prioridades} & \textbf{Descrição}\tabularnewline
\hline 
\hline 
1 & Receber e mostrar dados e estatísticas de monitoramento. \tabularnewline
\hline 
2 & Permitir o controle dos aparelhos automatizados. \tabularnewline
\hline 
3 & Deve permitir o controle e visualização do sistema de segurança. \tabularnewline
\hline 
4 & Deve mostrar quando algum componente estiver defeituoso. \tabularnewline
\hline 
\end{tabular}

\section{Pesquisa realizada sobre pacotes de automação}
\begin{tabular}{|c|c|c|c|c|c|c|}
\hline 
\textbf{Nome} & \textbf{Segurança} & \textbf{Iluminação} & \textbf{Irrigação} & \textbf{Multimídia} & \textbf{Climatização} & \textbf{Câmera}\tabularnewline
\hline 
\hline 
Gdsautomacao & x & x & x & x & x & x\tabularnewline
\hline 
Cynthron & x & x & - & x & - & -\tabularnewline
\hline 
Smart home & x & x & - & x & x & x\tabularnewline
\hline 
Automatedliving & x & x & ? & x & x & ?\tabularnewline
\hline 
Homeseer & x & x & x & x & x & x\tabularnewline
\hline 
Control4 & x & x & - & - & x & x\tabularnewline
\hline 
Creston & x & x & x & x & x & x\tabularnewline
\hline 
Vera & x & x & x & - & x & x\tabularnewline
\hline 
Stables conneted & x & x & x & - & x & x\tabularnewline
\hline 
Iris & x & x & x & - & x & x\tabularnewline
\hline 
Savant & x & x & x & x & x & x\tabularnewline
\hline 
SmartThings & x & x & - & - & Third Party & Third Party\tabularnewline
\hline 
Nexia & x & x & - & - & x & x\tabularnewline
\hline 
Wallpad & - & x & x & x & & -\tabularnewline
\hline 
AutomaticHouse & x & x & x & x & x & -\tabularnewline
\hline 
\end{tabular}







%Sistema de Automação para Casa Sustentável
%
%Versão 2.1.1
% 
%Histórico da Revisão
%Data
%Versão
%Descrição
%Autor
%28/09/2015
%1.0.0
%Criação
%Lucas V. T. Brilhante
%28/09/2015
%1.1.0
%Adição da seção 1. Introdução
%Lucas V. T. Brilhante
%28/09/2015
%1.2.0
%Adição da seção 2. Posicionamento
%Mateus M. F. Mendonça
%28/09/2015
%1.3.0
%Adição da seção 3. Descrição dos Envolvidos e Usuários
%Lucas V. T. Brilhante
%28/09/2015
%1.4.0
%Adição da seção 4. Ambiente do Usuário
%Mateus M. F. Mendonça
%28/09/2015
%1.5.0
%Adição da seção 5. Visão Geral do Sistema
%Mateus M. F. Mendonça
%28/09/2015
%1.6.0
%Adição da seção 6. Requisitos Funcionais do Sistema
%Vitor N. A. Ribeiro
%28/09/2015
%1.7.0
%Adição da seção 7. Requisitos Não Funcionais do Sistema
%Vitor N. A. Ribeiro
%28/09/2015
%1.8.0
%Adição da seção 8. Procedências e prioridades.
%Lucas V. T. Brilhante
%29/09/2015
%1.9.0
%Adição da seção 9. Tabela comparativa
%Guilherme S. Lionço
%29/09/2015
%2.0.0
%Revisão e finalização do documento
%Vitor N. A. Ribeiro, 
%Marcelo M. Oliveira,,
%Flávio da C. Paixão
%29/09/2015
%2.1.0
%Adição da seção 10. Referências Bibliográficas
%Vitor N. A. Ribeiro
%04/10/2015
%2.1.1
%Revisão
%Vitor N. A. Ribeiro
%
%
%Índice Analítico
%1.      Introdução                    3
%1.1    Finalidade                   3
%1.2  Escopo                3
%1.3  Visão Geral                3
%1.4  Acrônomos                3
%2.      Posicionamento                 4
%2.1.    Descrição do Problema            4
%2.2.    Sentença de Posição do Sistema          4
%3.      Descrições dos Envolvidos e Usuários              5
%3.1.    Resumo dos Envolvidos              5
%3.2.    Resumo dos Usuários              6
%3.3  Perfis dos Envolvidos              6
%4.      Ambiente do Usuário                 9
%5.      Visão Geral do Sistema              9
%6.1.    Perspectiva do Sistema             9
%6.2.    Suposições e Dependências               9
%6.3  Licenciamento e Instalação            9
%6.      Requisitos Funcionais do Sistema            10
%  6.1  Tabela de Requisitos              10
%7.  Requisitos Não Funcionai do Sistema          11
%  7.1  Tabela de Requisitos Não Funcionais          11
%7.      Procedências e Prioridades                 12
%9.  Pesquisa realizada sobre pacotes de automação        12
%10.  Referências Bibliográficas              14
%
%Visão
% 
%Introdução
%O documento de visão busca coletar, analisar e definir necessidades e recursos de nível superior do “Sistema de Automação para Casa Sustentável”. Concentra-se nos recursos necessários, usuários-alvo e nas razões do projeto. Os detalhes de como o SACS satisfaz essas necessidades são descritos nos casos de uso e nas especificações suplementares.
%
%1.1 – Finalidade
%Este documento tem como objetivo apresentar, analisar e definir as necessidades e características do SACS. O SACS tem como objetivo controlar e monitorar todo o sistema do Projeto “Casa Sustentável”. O documento descreverá as características gerais do sistema, os envolvidos em seu desenvolvimento e o público alvo. 
%1.2 - Escopo
%O SACS, permite o usuário monitorar gasto de energia elétrica e consumo de água, controlar o sistema de segurança, visualizar a imagem das câmeras de segurança e controlar todo o sistema automatizado, incluindo sistema de jardinagem, luzes internas e outros aparelhos internos. O sistema também deve permitir a visualização da produção de energia das placas de energia solar e dos motores eólicos, além de mostrar o nível de reservatório de água da chuva.
%1.3 - Visão geral
%  Faremos uso de nosso documento como uma forma de organizar informações necessárias para futuras atividades do projeto em desenvolvimento. Informações a cerca do contexto do projeto, funcionalidades, e características serão organizados e estarão disponíveis neste documento afim de poupar qual vir a atrasar o processo de desenvolvimento do projeto. Todo e qualquer tempo será poupado com o uso deste documento de visão.
%1.4 - Acrônimos
%SACS - Sistema de Automação para Casa Sustentável
%IHC - Interação humano computador
%HVAC - Heating, ventilating, and air conditioning
%MVC - Model-View-Controller
%RF - Requisitos funcionais
%RNF - Requisios não funcionais
%GPS - Global Position System
%Z-WAVE - É uma tecnologia desenvolvida especialmente para automação residencial por uma empresa chamada Zensys. É um protocolo de comunicação completamente sem fios que usa uma largura da banda estreita para enviar comandos de controlo e, potencialmente, dados secundários .
%
%
%
%2.      Posicionamento
%Descrição do Problema
% 
%O problema
%Valor elevado do consumo desnecessário de energia elétrica
%Afeta
%Pessoas que moram em residências, sejam casas ou apartamentos, e fazem uso da energia elétrica
%Cujo impacto é
%Alto custo da conta de energia elétrica
%Uma boa solução seria
%Desenvolvimento de um sistema de automação capaz de controlar o consumo da energia elétrica de acordo com o requisitos.
% 
%2.   Sentença de Posição do Sistema
%Para
%Cidadãos brasileiros residentes na Casa Sustentável e inteligente 
%Que
%Se importam com o alto consumo desnecessário de energia elétrica e desejam maior conforto
%O SACS
%É um software ou um conjuntos de softwares de transmissão e controle de dados que integra sensores e age de forma inteligente
%Que
%Garante conforto para os residentes e economia de energia
%Diferente de
%[Crestron](http://www.crestron.com/markets/home_theater_and_whole_house_home_automation/)
%Nosso sistema
%Integra sensores de forma inteligente para controle do consumo de energia elétrica, armazenamento e reuso de água não potável
%
%
%
%
%
%
%
%3.      Descrições dos Envolvidos e Usuários
%3.1  Resumo dos Envolvidos
%Nome
%Descrição
%Responsabilidades
%Professor das disciplina de Projeto Integrador 1.
%É responsável por ministrar aulas de projeto integrador 1, direcionando caminhos para o projeto.
%
%
%Mantem o foco do projeto ministrando aulas sobre gerência de projetos, direcionando o caminho a ser seguido pelo projeto.
%Alunos de Projeto Integrador 1.
%São os alunos que cursam a disciplina de Projeto Integrador 1.
%Iram desenvolver o projeto, especificando quais tecnologias serão usadas, que se encaixam melhor no escopo do projeto, visando a sustentabilidade.
%Aluno Gerente de Projeto Integrador 1.
%É o aluno que cursa a disciplina de projeto Integrador e assumiu a posição de gerente.
%Irá garantir que todos do grupo estão desempenhando sua função e deverá integrar todas as partes e passar para o gerente do projeto “Casa Sustentável”.
%Desenvolvedores e Gerentes
%Profissionais da área que implementarão e integrarão o projeto já pronto dos alunos de PI1.
%Irão produzir o sistema visionado e projetado na matéria de PI1.
%Moradores da “Casa Sustentável”
%Usuário das funcionalidades da “Casa Sustentável”
%Possuir fundo financeiro necessário para investir no projeto da “casa sustentável”
% 
%3.2  Resumo dos Usuários
%Nome
%Descrição
%Moradores da “Casa Sustentável”
%Pessoa Física que adquiriu o produto “Casa Sustentável”, especificado no projeto da disciplina de PI1.
%
%
%
%3.3  Perfis dos Envolvidos
%-----_     3.1 Equipe projetista da disciplina de PI1
%Representantes
%Arnoldo T. M. Lima
%Flavio C. Paixao
%Itiane T. B. Almeida
%Lucas V. T. Brilhante
%Marcelo M. Oliveira
%Talyta V. Cabral
%Victor B. Batalha
%Gabriel S. Soares
%Vitor N. A. Ribeiro
%Descrição
%Farão o documento do projeto com todas as especificações técnicas necessárias.
%Tipo
%Estudantes da Universidade de Brasília, Campus do Gama da matéria de Projeto Integrador 1, que pesquisaram todas as especificações do projeto e desenvolveram usando todos os conhecimentos de projeto passado pelos professores.
%Responsabilidades
%Trabalhar em conjunto para montar um documento que especifique todos os aspectos do projeto. Pesquisar conhecimentos técnicos a serem aplicados no projeto.
%Critérios de Sucesso
%Gerir de maneira adequada, trabalhar em conjunto, desenvolver em equipe adequando ao tema do projeto, custo e prazo definido pela matéria.
%Envolvimento
%Alto.
%
%3.2 Gerente de automação do projeto “Casa Sustentável”
%Representantes
%Gabriel S. Soares
%Descrição
%Aluno da matéria de PI1 que é gerente da área de automação.
%Tipo
%Aluno habilitado e disponível, capaz de gerir a equipe de projeto de automação.
%Responsabilidades
%Garantir que todos os membros estão desempenhando sua função e entregando o que foi decidido. Deve também organizar as entregas de forma a integrar os assuntos e garantir a qualidade do projeto.
%Critérios de Sucesso
%Gerir corretamente o tempo de forma a abranger a diferentes variáveis que podem atrasar a entrega.
%Envolvimento
%Alto.
% 
%
%3.3 Principais Necessidades dos Usuários ou dos Envolvidos
%Necessidade
%Prioridade
%Preocupações
%Solução Atual
%Solução Proposta
%Visualizar o gasto e a produção energética da casa.
%Alta
%Obter dados energéticos para visualização do balanço, já que, o objetivo do projeto “Casa sustentável” ter ser sustentável, isto é, inclui gerar sua própria energia.
%Não há solução.
%Receber dados vindo da smart grid e dos geradores de energia e mostrar para o usuário através de gráficos.
%Controlar aparelhos domésticos com a automação implementada.
%Alta
%O usuário tem que ter a habilidade de controlar todos os componentes automatizados da casa.
%Há vários softwares no mercado que fazem esse tipo de monitoramento no mercado, mas nenhum deles tem integração com todos os sistemas de monitoramento sustentável e de geração de energia que são necessários.
%Desenvolver um sistema embarcado que consiga receber todos os dados de sensores e consiga controlar os aparelhos com automação.
%Acessibilidade 
%Alta
%O software deve ter uma interface intuitiva e fácil de usar, já que o software engloba muitas funções de monitoramento e controle.
%Busca de técnicas de programação e criação de interface simples.
%Analisar softwares presentes no mercado e fazer trabalho de pesquisa com protótipos com usuários reais.
% 
%
%
%4.      Ambiente do Usuário
%O sistema será implementado e testado em um servidor Linux conectado através de Z-Waves juntamente ao que será o SACS.
%As restrições de uso incluem a compra do dispositivo com o software embutido e conexão com a internet na casa.
%
%5.      Visão Geral do Sistema
%5.1  Perspectiva do Sistema
%O sistema tem como objetivo automatizar residências com a finalidade de trabalhar com termostatos, fechaduras, sistema de segurança, equipamentos de áudio e video, sistema de irrigação, portas de garagem, sensores ambientais, câmeras, luzes, gerenciamento de energia, internet e redes relacionadas, controlador de piscinas e spa, integração com radio frequência, integrado com robôs de limpeza, segurança dos dados, integração com telefones, fax, emails e possuir interface de usuário.
%5.2  Suposições e Dependências
%Para a utilização do sistema é suposto que a casa possua conexão com internet e vários sensores para controle do gasto de energia e sensor de fluxo de água.
%Caso haja necessidade de alterações no projeto, as alterações deverão ser feitas neste documento traduzindo de melhor maneira possível.
%5.3  Licenciamento e Instalação
%Alguns recursos explorados no desenvolvimento do sistema poderá ser necessário a compra de licenças.
%
%6.      Requisitos Funcionais do Sistema
%Esta seção tem a função de informar de forma resumida todas as características e funcionalidades do sistema.
%6.1 Tabela de Requisitos
%Requisito
%Descrição
%Prioridade
%Dependência
%RF01
%O sistema deve possuir uma rede inteligente que através de informações coletadas, retorne aos usuários dados dos gastos da casa.
%Alta
%
%
%RF02
%O sistema deve possuir métodos e algoritmos de processamento de sinais
%Médio
%
%
%RF03
%O sistema deve fazer a gestão da informação.
%Médio
%
%
%RF04
%O sistema deve garantir automatização da iluminação, multimídia, cortinas, HVAC, banho, jardim e segurança.
%Médio
%RNF01, RNF05
%RF05
%O sistema deve ter capacidade de detectar presença.
%
%
%
%
%RF06
%O sistema deve ter capacidade de controlar posição de cortinas.
%
%
%
%
%RF07
%O sistema deve ter capacidade de controlar o fluxo de água.
%
%
%
%
%RF08
%O sistema deve ter capacidade de medir umidade do ambiente.
%
%
%
%
%RF09
%O sistema deve ter capacidade de medir umidade do solo.
%
%
%
%
%RF10
%O sistema deve ter capacidade de medir a temperatura da água.
%
%
%
%
%RF11
%O sistema deve utilizar sensores de movimentos.
%
%
%
%
%
%
%
%
%7.      Requisitos Não Funcionais do Sistema
%Resumo em alto nível de outras características do sistema, tipicamente não funcionais.
%7.1 Tabela de Requisitos Não Funcionais
%Identificador
%Recurso
%RNF01
%O sistema deve ser passível de cálculo da taxa de transmissão de dados
%RNF02
%O sistema de segurança deve ter capacidade de se manter em funcionamento e estável durante situação críticas.
%RNF03
%Os dados do sistema deverá ter garantia de integridade, anonimato e consistência.
%RNF04
%Deve ser garantido o monitoramento dos dados a partir da internet.
%RNF05
%Os sensores devem ser passíveis de aferir sua precisão.
%RNF06
%O sistema de IHC deve ser de fácil usabilidade.
%RNF07
%O sistema de IHC deve ser intuitivo.
%RNF08
%O sistema de IHC deverá ser passível de uma manutenção não complexa.
%RNF09
%O sistema deve possuir uma unidade central de processamento.
%RNF10
%O sistema deve fazer o monitoramento em tempo real.
%RNF11
%O sistema deve atender as metas de usabilidade que permita a visualização e manipulação dos componentes da casa.
%
%8.   Procedência e Prioridades
%
%Prioridades
%Descrição
%1
%Receber e mostrar dados e estatísticas de monitoramento.
%2
%Permitir o controle dos aparelhos automatizados.
%3
%Deve permitir o controle e visualização do sistema de segurança.
%4
%Deve mostrar quando algum componente estiver defeituoso.
%
%9.   Pesquisa realizada sobre pacotes de automação
%
%Nome
%Segurança
%iluminação
%irrigação
%multimídia
%climatização
%câmera
%Gdsautomacao
%x
%x
%x
%x
%x
%x
%Cynthron
%x
%x
%-
%x
%-
%-
%Smart home
%x
%x
%-
%x
%x
%x
%Automatedliving
%x
%x
%?
%x
%x
%?
%Homeseer
%x
%x
%x
%x
%x
%x
%Control4
%x
%x
%-
%-
%x
%x
%Creston
%x
%x
%x
%x
%x
%x
%Vera
%x
%x
%x
%-
%x
%x
%Stables conneted
%x
%x
%x
%-
%x
%x
%Iris
%x
%x
%x
%-
%x
%x
%Savant
%x
%x
%x
%x
%x
%x
%SmartThings
%x
%x
%x
%-
%Third party
%Third party
%Nexia
%x
%x
%-
%-
%x
%x
%Wallpad
%-
%x
%-
%x
%
%
%-
%AutomaticHouse
%x
%x
%x
%x
%x
%-
%
%
%
%Portão
%Controle
%Energia
%Janelas
%Biometria
%Garagem
%Sensores de Entrada
%Outras features
%Cenas
%Rede de Dados
%x
%-
%-
%-
%-
%-
%-
%-
%-
%-
%-
%mobile, voz
%-
%
%
%-
%-
%-
%-
%-
%-
%-
%mobile
%-
%-
%-
%-
%-
%sensor de vasamento de água
%-
%-
%?
%-
%-
%-
%-
%-
%-
%-
%-
%-
%x
%Mobile
%x
%x
%
%
%x
%x
%meteorológia
%x
%
%
%
%
%Mobile
%-
%-
%-
%-
%x
%-
%-
%-
%x
%Mobile
%x
%x
%x
%x
%Third party
%-
%-
%-
%x
%Mobile
%x
%Third party
%x
%Third party
%x
%-
%-
%-
%x
%Mobile
%x
%x
%x
%x
%x
%-
%-
%-
%x
%Mobile
%x
%Third party
%x
%x
%x
%-
%-
%-
%-
%Mobile
%x
%x
%x
%-
%-
%-
%-
%-
%x
%Mobile
%x
%Third party
%x
%Third party
%x
%-
%-
%-
%x
%Mobile
%x
%-
%-
%-
%x
%-
%-
%-
%-
%Mobile
%-
%-
%-
%-
%-
%-
%-
%-
%-
%x
%x
%-
%-
%-
%x
%telefone
%x
%x
%
%
%
%
%
%10.      Referências Bibliográficas
%  
%Tabelas com os links referentes aos produtos relacionados. Disponível em: <https://docs.google.com/spreadsheets/d/1WPOmOh2WVWy80Mfl5BgaIS0yhlrtesQmxYhjWWiehKQ/edit#gid=0>. Acessado em: 29 de Setembro de 2015.
%


\chapter{Especificação de Requisitos}


Casa Sustentável

Especificação de Requisitos de Software
do Sistema de Automação para Casa Sustentável (SACS)
 
 
Versão 1.1.3
 
 


Histórico da Revisão
Histórico da Revisão
Data
Versão
Descrição
Autor
02/10/2015
1.0.0
Criação
Vitor N. A. Ribeiro
02/10/2015
1.1.0
Adição da seção 1. Introdução
Vitor N. A. Ribeiro
02/10/2015
1.1.1
Adição da seção 3.1. Requisitos Funcionais
Marcelo M. Oliveira,
Ricardo L. ,
Vitor N. A. Ribeiro
03/10/2015
1.1.2
Adição da seção 3.9. Interfaces
Lucas V. T. Brilhante
04/10/2015
1.1.3
Adição da seção 3.2. Requisitos Não Funcionais
Flávio da C. Paixão
05/10/2015
1.1.4
Dividindo e incrementando seção 3.1.4
Mateus M. F. Mendonça
06/10/2015
1,1,5
Especificação de requisitos
Lucas V. T. Brilhante
Flávio da C. Paixão
Mateus M. F. Mendonça
Vitor N. A. Ribeiro
06/10/2015
1,1,6
Formatação
Lucas V. T. Brilhante


















Índice Analítico
1. Introdução         
1.1 Finalidade     
1.2 Escopo     
1.3 Definições, Acrônimos e Abreviações     
1.4 Referências     
1.5 Visão Geral     
2. Descrição Geral    
3. Requisitos Específicos
3.1 Requisitos Funcionais
3.1.1 <Requisito Funcional Um>        
3.2 Requisitos Não Funcionais
3.2.1 <Requisito Não Funcional Um>  
3.8 Componentes Comprados     
3.9 Interfaces     
3.9.1 Interfaces de Usuário           
3.9.2 Interfaces de Hardware           
3.9.3 Interfaces de Software           
3.9.4 Interfaces de Comunicação           
3.10 Requisitos de Licenciamento     
3.11 Observações Legais, de Direitos Autorais etc     
3.12 Padrões Aplicáveis     
4. Informações de Suporte    









Especificação de Requisitos de Software
1.                  Introdução

1.1     Finalidade
    
Este documento de Especificação de Requisitos de Software concentra-se na coleta e na organização de todos os requisitos que envolvem o SACS. Descrevendo todos os requisitos de software para cada característica e funcionalidade de uma determinada ação do sistema.

1.2     Escopo
    
    O SACS, permite o usuário monitorar gastos de energia elétrica e consumo de água, controlar o sistema de segurança, visualizar a imagem das câmeras de segurança e controlar todo o sistema automatizado, incluindo sistema de jardinagem, luzes internas e outros aparelhos internos. O sistema também deve permitir a visualização da produção de energia das placas de energia solar e dos motores eólicos, além de mostrar o nível de reservatório de água da chuva.

1.3     Definições, Acrônimos e Abreviações
    
    SACS - Sistema de Automação para Casa Sustentável

1.4     Referências
    

1.5     Visão Geral
    Este documento servirá como base para a implantação do sistema SACS durante todo o desenvolvimento do projeto da Casa Sustentável. Informações acerca das funcionalidades e características do SACS serão organizados e estarão disponível neste documento afim de garantir que todas as necessidades sejam implementadas.
    Esse documento está dividido em: Descriçao Geral, Lista de Requisitos e Requisitos Específicos.


2.                  Descrição Geral

    O sistema SACS lidará com termostatos, fechaduras, sistema de segurança, equipamentos de áudio e video, sistema de irrigação, portas de garagem, sensores ambientais, câmeras, luzes, gerenciamento de energia, internet e redes relacionadas, controlador de piscinas e spa, integração com radio frequência, integração com robôs de limpeza, segurança dos dados, integração com telefones, fax e emails, possuirá interface de usuário, integração a consoles de jogos entre outros utilitários.

    Para a perfeita utilização do SACS é essencial que o sistema esteja em constante conectividade com a internet. Será necessário instalar sensores nas entradas das tomadas de energia para o controle dos gastos. A instalação de um núcleo de dados onde seja refrigerada e 
isolada se possível. 

3.  Requisitos Específicos

3.1     Requisitos Funcionais

3.1.01    Monitoramento energético.

3.1.02    Ambientação.

3.1.03    Gestão da informação.

3.1.04 Controle da Disposissão das Cortinas

3.1.05    Automatização de multimídia.

3.1.06    Automatização Hidrico.

3.1.07    Automatização do jardim.

3.1.08    Segurança

3.1.09 Detectar presença
3.1.10    O sistema deve ter capacidade de medir umidade.

    


3.2              Requisitos Não Funcionais
    
3.2.01    Taxa de transmição de dados
3.2.02    Estabilidade do sistema
3.2.03    Integridade dos dados
3.2.04    Monitoramento atráves da internet.
3.2.05    Precisão dos dados dos sensores.
3.2.06    Usabilidade do sistema.
3.2.07    Unidade central de processamento.
3.2.08    Monitoramento em tempo real.

3.8     Componentes Comprados
    
O sistema possuirá dispositivos de hardware e software, que iram auxiliar nas funcionalidades de automatização da SACS, e que serão adquiridos pagando as devidas taxas referentes ao licenciamento e operabilidade destes dispositivos.



3.9     Interfaces


Número de ordem
Nome
Ator
Caso de uso
Descrição
1         
Tela de monitoramento
 Usuário
Monitorar dados da smart grid.
Visualizar dados da smart grid a partir da tela de monitoramento. 
2         
Tela de monitoramento de sensores
Usuário
Monitorar dados de sensores
Visualizar dados dos sensores na tela de monitoramento.
3         
tela de controle
Usuário
Controle multimídia
Acionar aparelhos multimídia conectados.
4         
 Tela de segurança
Usuário 
Controle de segurança 
Visualizar status do equipamento de segurança e ligar e desligar sistemas. 
5         
Tela de segurança 
Usuário 
Visualizar imagem de cameras 
Obter imagens na tela do celular a partir das cameras de segurança. 
6         
Tela de jardinagem 
Usuário 
Controlar e visualizar status de jardinagem 
Ativar/desativar irrigadores e controles de jardim. 










5 Requisitos Funcionais

5.1 Monitoramento energético
5.1.1 Descrição
O sistema deve medir o consumo energético em tempo real e gerar relatórios para o usuário.

5.1.2 Entradas
Gastos de energia gerado por cada ponto de acesso de energia (tomada) da casa.

5.1.3 Saídas
Deve ser gerado relatórios diários, semanais e mensais, de gastos energéticos,
de acordo com a requisição do usuário.

5.1.4 Processamento
Todo o consumo de energia gerado pela casa deve ser armazenado, de forma a ser acessível local e remotamente.
Todo o consumo de energia gerado pela casa deve ser verificado, antes de
armazenado.

5.1.5 Restrições
N/A

5.1.6 Manipulação de erros
Todos os dados do consumo energético deve ser verificado, e caso haja inconsistência ou erros nos dados, deve ser registrado no log do



5.2 Ambientação
5.2.1 Descrição
As mudanças de temperatura e luzes devem influenciar na abertura das persianas, na força das luminárias e no HVAC para que o ambiente seja mantida constante e agradável.

5.2.2 Entradas
Sinais de sensores de luz e temperatura.

5.2.3 Saídas
Deve ser gerado um controle a fim de optimizar luz e temperatura fornecida pela casa.

5.2.4 Processamento
A luz e a temperatura da casa deve ser verificada e as modificações necessárias devem ser requisitadas a fim de manter o ambiente.

5.2.5 Restrições
N/A

5.2.6 Manipulação de erros
Caso os dados não estejam acurados ou divergentes do esperados o sistema deve avisar o erro e não manipular mais nos objetos de automação de luz e temperatura.

5.4 Controle da Disposissão das Cortinas
5.4.1 Descrição
 O sistema a partir de um controle remoto ou ambiente pré configurado, deverá mudar a disposição das cortinas.

5.4.2 Entradas
    Comandos enviados de um controle remoto.

5.4.3 Saídas
    Manipulação da disposição das cortinas de acordo com o comando.

5.4.4 Processamento
    A partir de dados pré-mapeados é feito o controle da disposição das cortinas referente ao que foi configurado.

5.4.5 Restrições
N/A

5.4.6 Manipulação de erro
    O sistema pode se deparar com a perda de um comando enviado a partir de um controle sendo necessário re-enviar o comando.


5.5 Controle multimídia
5.5.1 Descrição
 O sistema deve garantir automatização de multimídia.

5.5.2 Entradas
    Comandos enviados de um controle remoto.

5.5.3 Saídas
    Manipulação de audio, TV, projetor, dentre outros sistema multimídia presentes na casa.

5.5.4 Processamento
    A partir de dados pré-mapeados é feito o controle de dispositivos de multimídia referente ao que foi configurado.

5.5.5 Restrições
N/A

5.5.6 Manipulação de erro
    O sistema pode se deparar com a perda de um comando enviado a partir de um controle sendo necessário re-enviar o comando.



5.6 Monitoramento hídrico
5.6.1 Descrição
O sistema deve medir o consumo hídrico em tempo real e gerar relatórios para o usuário.

5.6.2 Entradas
Dados de consumo hídrico em tempo real.

5.6.3 Saídas
Relatórios diários, semanais e mensais, de gastos hídricos, de acordo com a requisição do usuário.

5.6.4 Processamento
Todo o consumo hídrico gerado pela casa deve ser verificado, antes de armazenado.
Os dados do consumo hídrico devem ser processados para gerar um relatório contendo o consumo e o custo estimado da conta de água.

5.6.5 Restrições
N/A

5.6.6 Manipulação de erros
Todo o consumo hídrico gerado pela casa deve ser armazenado, após verificado, de forma segura em um hardware local.




5.7 Controle de Jardim
5.7.1 Descrição
O sistema deve garantir automatização do jardim.

5.7.2 Entradas
Irrigação do jardim,de acordo com a necessidade de cada planta.

5.7.3 Saídas
Ser otimizado de modo a economizar água sem desperdicios.

5.7.4 Processamento
Ser colocado um tempo para irrigação de cada planta,vendo a necessidade de cada uma.

5.7.5 Restrições
N/A

5.7.6 Manipulação de erros
Caso haja algum vazamento ou desperdicio de água o sistema deve avisar para não haver perdas de água desnecessarias.






5.8 Controle da segurança de dados
5.8.1 Descrição
Os dados do sistema deverá garantir integridade, anonimato e consistência.

5.8.2 Entradas
    Todos os dados do fluxo do sistema de forma

5.8.3 Saídas
    Dados criptografados e com seu respectivo hash.

5.8.4 Processamento
    A partir dos dados de entrada, o sistema passa eles por dois algoritmos, um que criptografa e outros que tira sua hash (serve como indentidade).

5.8.5 Restrições
N/A

5.8.6 Manipulação de erro
    Todo dado após criptografado ao chegar ao destino é calculado novamente o hash e comparado com o anterior, antes de navegar pela rede e caso confirmado é de criptografado.


5.9 Detectar Presença
5.9.1 Descrição
 O sistema a partir de sensores de presença deverá identificar qualquer movimento fora e dentro da casa.

5.9.2 Entradas
    Alertas de detectação de movimento no perimetro.

5.9.3 Saídas
    Informação da localização de onde foi detectado movimento e projeção da camera de video mais próximo deste ponto.

5.9.4 Processamento
    Após a detecção de movimento, o sensor enviar para o sistema informações de onde foi detectado e libera a imagem da câmera mais próximo do local identificado.

5.9.5 Restrições
N/A

5.9.6 Manipulação de erro
O sistema deverá lidar com a capacidade de ignorar movimentos de animais, ventos, folhas entre outros objetos.


5.10 Monitoramento do solo
5.10.1 Descrição
O sistema deve medir a umidade do solo em tempo real, gerar relatórios para o usuário e 

5.10.2 Entradas
Dados da umidade do solo em tempo real.

5.10.3 Saídas
Relatórios diários, semanais e mensais, da umidade do solo, de acordo com a requisição do usuário.

5.10.4 Processamento
Os dados da umidade do solo devem ser armazenados de forma segura em um hardware local, e processados para gerar um relatório contendo a variação ao longo do dia.
O usuario deve ser alertado caso a umidade atinja um nível crítico.

5.10.5 Restrições
N/A

5.10.6 Manipulação de erros
Todos os dados da medição da umidade do solo deve ser verificado, antes de armazenado.



















6. REQUISITOS NAO FUNCIONAIS
6.1 Taxa de transmição de dados
6.1.1 Performance 
O sistema deve ser passível de cálculo da taxa de transmissão de dados de alta velocidade.
6.1.2 Confiabilidade
    O sistema deve manter uma taxa de transmição de dados altas sem falha.
6.1.3 Disponibilidade
    O sistema deve manter uma taxa de transmição de dados alta durante todo o periodo de funcionamento.

6.1.4 Segurança
    A transmissão de dados deve ser inacessível a dispositivos não autorizados ou cadastrados.
6.1.5 Manutenibilidade
    Em caso de falha, o sistema deve ser capaz de se recuperar em um tempo minimo a fim de nao deixar os aparelhos desamparados da trasmissão de dados.

6.1.6 Portabilidade
    A transmissão de dados de alta velocidade deve abranger todos os dispositivos desejados da casa, além de ser capaz de se expandir para que a transmissão alcance novos dispositivos.






 O sistema de segurança deve ter capacidade de se manter em funcionamento e estável durante situação críticas.
    
    Informações: Segurança

    Regra: O sistema deve possuir dispositivos que possam manter estável o funcionamento energetico da casa.


6.2 Estabilidade sistemática
6.2.1 Performance 
O sistema deve se manter estável durante situações críticas
6.2.2 Confiabilidade
    O sistema deve se manter estável afim de manter o funcionamento energético da casa.
6.2.3 Disponibilidade
    O funcionamento sistema deve ser manter estável 24hrs (horas) por dia.
6.2.4 Segurança
    O sistema não deve ser vulnerável a falhas internas e externas que prejudiquem seu funcionamento por tempo integral.
6.2.5 Manutenibilidade
    O sistema deve possuir uma manutenção pouco complexa e que não limite a eficiência e a eficâcia da mesma caso apresente alguma falha no seu funcionamento.
6.2.6 Portabilidade
    O sistema deve estar acessível a todos os dispositivos que regulem e mantenham o funcionamento estável do mesmo.


6.3 Integridade do sistema
6.3.1 Performance 
   Os dados do sistema deverá ter garantia de integridade, anonimato e consistência. 
6.3.2 Confiabilidade
    O sistema deve ser capaz de garantir a segurança de dados da casa garantindo que nenhum dado será apagado ou acessado.
6.3.3 Disponibilidade
    O sistema deve manter a integridade dos dados durante todo o tempo de funcionamento.
6.1.4 Segurança
    Os dados do sistema devem ser inacessíveis a dispositivos ou usuários não autorizados, preferencialmente atravez de criptografia dos dados e requerimento de login e senha do usuário.
6.3.5 Manutenibilidade
    Em caso de falha, o sistema deve ser capaz de se recuperar em um tempo minimo a fim de nao deixar os aparelhos desamparados da trasmissão de dados.

6.3.6 Portabilidade
A integridade dos dados deve se manter intácta para qualquer aparelho.


6.4 Monitoramento atráves da internet
6.4.1 Performance 
Deve ser garantido o monitoramento dos dados a partir da internet
6.4.2 Confiabilidade
    O sistema deve manter uma transmição continua dos dados de segurança sempre que solicitado.  
6.4.3 Disponibilidade
    O sistema deve ter disponivel os dados sempre que solicitado.

6.4.4 Segurança
    A transmissão de dados deve ser inacessível a dispositivos não autorizados.
6.4.5 Manutenibilidade
    Em caso de falha, o sistema deve ser capaz de se recuperar em um tempo minimo a fim de não afetar os dados requeridos pelo usuário.

6.4.6 Portabilidade
    O sistema deve ser acessivel de qualquer aparelho cadastrado.


3.2.05    Os sensores devem ser passíveis de aferir sua precisão.
    
    Informações: Verificação dos sensores

    Regra: O sistema deve possuir uma funcionalidade que possa checar a precisão dos sensores presentes na casa.



6.5 Alta precisão nos dados dos sensores.
6.5.1 Performance 
   Os dados vindos dos sensores devem ser de alta precisão e velocidade, pois a partir dele, ações aconteceram no sistema. 
6.5.2 Confiabilidade
    O sistema de sensores devem ter alta confiabilidade nos seus dados, pois a partir deles ações seram tomadas externamente.
6.5.3 Disponibilidade
    O monitoramento de dados depende diretamente da disponibilidade do sistema, sendo assim, o sensor deve ter um alto nivel de disponibilidade para fornecer esses dados sem comprometer o sistema.
6.5.4 Segurança
    Os dados enviados pelos sensores devem ser mantidos dentro da rede segura e criptografada do sistema, impedindo monitoramento externo de terceiros.
6.5.5 Manutenibilidade
    Em caso de falha, o sistema deve ser de fácil manutenção, podendo se resolver a maioria das falhas com procedimentos simples, evitando necessidade de um técnico com frequência.

6.5.6 Portabilidade
    Os dados da casa devem ser acessiveis de qualquer aparelho conveniado com o sistema de qualquer lugar, atravez da internet.



3.2.06    O sistema deve ser de fácil usabilidade.
    
    Informações: Metas de usabilidade

    Regra: O sistema deve estar de acordo com as metas de usabilidade afim de torna-lo mais acessivel para os moradores da casa.
6.6 Usabilidade do Sistema
6.6.1 Performance 
    O sistema deve ser de facil usabilidade para ser de fácil manuseio.
    
6.6.2 Confiabilidade
O sistema deve manter sua interface disponivel sempre que o usiuário solicitar.

6.6.3 Disponibilidade
    O sistema deve ficar disponivel sempre que o usuário decidir manipular algo que exiga o uso do sistema.
6.6.4 Segurança
    
6.6.5 Manutenibilidade
    Em caso de falhas o sistema deve ser capaz de passar um feedback rapidamente para a empresa,para que seja feito a manutenção o mais rapido possivel para que o usuário não seja prejudicado.
    
6.6.6 Portabilidade
    





6.7 Central de processamento
6.7.1 Performace
    O sistema deve possuir uma CPU (unidade central de processamento) capaz de satisfazer todos os processos executáveis e significativos de entrada e saida desse sistema.
6.7.2 Confiabilidade
    A CPU deve ser capaz de realizar as instruções do sistema afim de executa-las de uma maneira completa, consistente e correta. 
6.7.3 Disponibilidade
    O funcionamento do sistema demanda da disponibilidade integral de sua CPU dado que ela é um componente importante para o funcionamento desse sistema.
6.7.4 Segurança
    A CPU deve manter-se estável caso seja imposta todo o fluxo de dados do sistema, afim de evitar falhas que comprometam o funcionamento do mesmo.
6.7.5 Manutenibilidade
    A manutenção da CPU deve ser feita com facilidade e segurança afim de minimizar a complexidade da mesma, caso esteja operando com dificuldades no processamento, e com uma alta temperatura.


3.2.09    O sistema deve possuir uma unidade central de processamento.
    
    Informações: Central de processamento.

    Regra: O sistema deve ser capaz de ser operado localmente atravês de uma central de processamento.

6.8 Monitoramento em tempo real
6.8.1 Performace
O sistema deve ser fornecer dados de monitoramento da casa em tempo real de processamento.    
6.8.2 Confiabilidade
    O sistema deve ser capaz de satisfazer o monitoramento em tempo real com o mínimo de interrupções possiveis, afim de maximizar a segurança da SACS.
6.8.3 Disponibilidade
    O sistema de processamento do monitoramento em tempo real deve estar disponivel em tempo integral por se tratar de uma funcionalidade significativa para a segurança da casa.
6.8.4 Segurança
    O sistema deve ser impassível a falhas de monitoramento, sendo estas externas e internas, que comprometam a segurança da casa.
6.8.5 Manutenibilidade
    O sistema deve possuir uma manutenção com o mínimo de complexidade para não comprometer a sua segurança, caso esteja apresentando um nível fora do normal de interrupções ou esteja apresentando falhas no seu funcionamento.
6.8.6 Portabilidade
    O sistema de monitoramento em tempo real deve estar disponível para ser visualizado por aplicações portadas para todos os dispositivos de visualização que o morador da casa tenha acesso.



3.8     Componentes Comprados
    
O sistema possuirá dispositivos de hardware e software, que iram auxiliar nas funcionalidades de automatização da SACS, e que serão adquiridos pagando as devidas taxas referentes ao licenciamento e operabilidade destes dispositivos.



\chapter{Aquisição de Software MPS.BR}



Objetivo da Aquisição
Justificativa
    Para o amplo conhecimento das possiveis soluções é necessário uma pesquisa de todas as soluções já feitas e analizar o custo do desenvolvimento de uma nova solução. Após esta pesquisa é necessário uma comparação.
A principio foram feitos os requisitos da solução (Especificação de Requisitos) e um documento de visão, que detalhava o pacote, que para generalização foi chamado de SACS (Sistema de automação para casa sustentável). A partir dessa analise de requisitos varias soluções deveriam ser apresentadas, mantendo em mente que o desenvolvimento a partir do zero do conjunto de software e hardware seria uma opção. Após a exaustiva busca pela internet por soluções prontas, uma tabela (Imagem\_01) foi feita comparando quais dos pacotes encontrados atendiam os requisitos especificados. Destes foi escolhido um que atendia todos os requisitos necessários do problema.
É necessário justificar porque este pacote escolhido teria um melhor custo-benefício do que um desenvolvido do zero. Observando os requisitos, seria necessário um conjunto de softwares, muitos deles sendo críticos (como sistema de segurança, REFERENCIA), levando assim a um desenvolvimento que seria muito longo e caro, o que tornaria o projeto inviável, além de não ser justificado pelo fato que as soluções separadas para uma casa sustentável são areas do conhecimento brandamente exploradas anteriormente por outros indivíduos e instituições. Portanto é possível chegar a conclusão que desenvolver a solução seria o uso de tempo e recursos que o projeto não prevê, além de estar ignorando o fato que já existem soluções amplamente testadas e aprovadas pela comunidade.


Objetivo
(REFERÊNCIA DOC. DE VISÃO)

Requisitos
        (REFERÊNCIA DOC. ESPECIFICAÇÃO DE REQUISITOS)





Termos Contratuais
Tipo de Contrato a ser Empregado
O contrato é estabelecido atráves da compra do pacote,onde o software está sendo disponibilizado, tendo seus termos de assinatura.
(Tipo de contrato a ser utilizado) 
Exemplo: Contrato de preço fixo, contrato de custos reembolsáveis ou contrato de preço unitário por ponto por função. 

Direitos de Distribuição, Uso e Propriedade do Software
            O software será 
(Estabelecimento dos direitos de distribuição, uso e propriedade do software, como, por exemplo, o número de cópias a serem distribuídas e a propriedade do código fonte, entre outros). 
Exemplo: O software desenvolvido estará sob uma licença de uso restrito ao contratante, protegidos por direitos autorais e de propriedade. A cópia, redistribuição, engenharia reversa e modificação do software proprietário são proibidas. Os programas de software serão de uso proprietário da organização cliente, inclusive seus códigos-fonte e documentação. A organização fornecedora não tem direito, disponibilidade ou qualquer outro tipo de participação em nenhum destes programas ou em qualquer cópia, modificação ou parte agregada de qualquer um destes programas. 

Lista de Software a ser Entregue

        (Lista dos S\&SC que devem ser entregues pelo fornecedor no final do contrato. Entre eles, devem ser considerados os serviços de suporte esperados do fornecedor). 
Exemplo: Os produtos a serem entregues ao final do contrato são: (i) o software instalado em seu ambiente de operação; (ii) os manuais do usuário, de instalação e do sistema; e (iii) os códigos-fonte  






Critérios de Aceitação do Software
Dentre os críterios de aceitação do HomeSeer  estão como causa  seu poder e seu apoio a uma variedade de protocolos de automação o software automatiza computadores dispositivos e sensores da casa, que podem incluir luzes, aparelhos, aparelhos de som, sensores de movimento, satisfazendo assim completamente os requisitos exigidos para a automação da casa. 
(Descrição de aspectos que devem ser satisfeitos para que o S\&SC sejam aceitos. Teoricamente todos os requisitos teriam que ser avaliados, o que nem sempre é prático. Estes são critérios que serão avaliados para apoiar o processo de aceitação. A garantia pode assegurar que os demais requisitos terão que ser atendidos durante o seu prazo de vigência). 

Seleção do Pacote


Imagem\_01
