\section{Introdução}

	A partir da integração das cinco engenharias\footnote{As cinco engenharias da UnB/FGA são Engenharia Aerospacial, Engenharia Automotiva, Engenharia Eletrônica, Engenharia de Energia e Engenharia de Software.} do campus, o projeto visa desenvolver um modelo de uma casa inteligente e autossustentável. Para atingir o êxito, se utilizará a metodologia ágil \textit{Scrum}, bem como ferramentas que auxiliaram o desenvolvimento do projeto\footnote{As ferramentas utilizadas no desenvolvimento do projeto são o \textit{Slack} como ferramenta de comunicação e \textit{Trello} como ferramenta de gestão de tarefas.}.

	A importância do projeto se dá devido à escassez e crise energética, que se identificada em âmbito local e global. Tal fato torna propício o desenvolvimento de projetos relativos à residências autossustetáveis, que visam a diminuição de custos a longo prazo, o aumento de conforto e segurança da residência.

	O local onde a casa será construída é no bairro Dahma localizado na cidade de Brasília, Distrito Federal, próximo à rodovia DF140. O prazo para o termino do projeto está marcada para o dia 25 de novembro de 2015. 